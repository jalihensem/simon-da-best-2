\documentclass[11pt, a4paper]{amsart}
\usepackage{amsmath}
\usepackage{amsthm}
\usepackage{amssymb}
\usepackage{parskip}
\theoremstyle{definition}
\newtheorem{problem}{Problem}
\newenvironment{solution}
{\begin{proof}[Solution]}
{\end{proof}}


\title{MAS116: Homework 5}
\author{Muhammad Ghazali bin Md Apandi}
\date{\today}

\begin{document}

\maketitle

\begin{problem}
	Let $f_n$ denote the $n$\emph{-th of Fibonacci number}. Thus, $f_1 = 1$, $f_2 = 1$ and $f_{n+2} = f_{n+1} + f_n$ for all $n \in \mathbb{N}$. Use induction to show that	
	
\[
	f_1^2 + f_2^2 + \ldots +f_n^2 = f_n f_{n+1}
\]

for all natural numbers $n \in \mathbb{N}$.
\end{problem}

\begin{solution}
	For each natural number $n \in \mathbb{N}$, let $P(n)$ be the statement $f_1^2 + f_2^2 + \ldots +f_n^2 = f_n f_{n+1}$.
	
	Firstly, we must check the base step for the induction. Hence, we can let $n = 1$. When $n = 1$, we have a statement of $P(1)$ where $1^2 = 1 \times 1$. Thus, $n = 1$ is true.
	
	Then, we can move to the next step which is inductive step. We assume that $P(k)$ is true for some $k \in \mathbb{N}$. Thus, 
	$$
	\sum_{i=1}^{k} f_i^2 = f_k f_{k+1}.
	$$
	
	From this, by assumption, we then have
	\begin{align*}
	\sum_{i=1}^{k+1} f_i^2 
	&= f_1^2 + f_2^2 + \ldots + f_k^2 + f_{k+1}^2 \quad\text{(by assumption)}\quad \\
	&= f_k f_{k+1} + f_{k+1}^2 \\
	&= f_{k+1}(f_k + f_{k+1})  \\
	&= f_{k+1} f_{k+2}
	\end{align*}
	
	Thus, $P(k+1)$ is true if and only if $P(k)$ is true.
	Hence, by induction, $P(n)$ is true for all $n \in \mathbb{N}$.
\end{solution}


\end{document} 
