\documentclass[11pt, a4]{amsart}
\author{Muhammad Ghazali bin Md Apandi}
\title{MAS116 : Presentation Homework 3}
\date{\today}
\usepackage{parskip}
\usepackage{hyperref}
\usepackage{amssymb}
\usepackage{amsthm}
\newtheorem{thm}{Theorem}[section]
\newtheorem{lem}[thm]{Lemma}
\theoremstyle{definition}
\newtheorem{defn}[thm]{Definition}

\begin{document}
\maketitle

\section{The square-root of 2}


We are going to investigate a solution of the
equation
\begin{equation}
\label{eq:root-2}
	x^2=2.
\end{equation}

The positive solution to equation(\ref{eq:root-2}) is denoted $\sqrt{2}$.
\begin{defn}
	The real number $\sqrt{2}$ is irrational.
\end{defn}

\begin{lem}
	Any rational number can be written in the form of $\frac{a}{b}$ with $a$ and $b$ as coprime integer.
\end{lem}

\begin{proof}
	Firstly, rational number can be written as $\frac{a}{b}$ where $a$,$b$ $\in \mathbb{Z}$ and $b \neq 0$ and $b > a$. Then, we can find the greatest common divisor(gcd) of both $a$ and $b$. From this, we can let gcd$(a,b) = c$ where $c$ is the greatest common divisor for both $a$ and $b$. This means that $a$ and $b$ are divisible by $c$. Hence, there exist integers $m$ and $n$ such that 
	\begin{align}
	\label{a}
	a = c \cdot m 
	\end{align}
	\begin{align}
	\label{b}
	b = c \cdot n
	\end{align}
	where $m,n \in \mathbb{Z}$.
	
	Then we can substitute \ref{a} and \ref{b} to write a new form of $\frac{a}{b}$ where
	\begin{equation}
	\frac{a}{b} = \frac{c.m}{c.n} = \frac{m}{n}.
	\end{equation}
	Since $c$ are gcd$(a,b)$, this implies that $m$ and $n$ has greatest common divisor of 1. Hence, this prove that $m$ and $n$ are coprime.
\end{proof}

\end{document}
